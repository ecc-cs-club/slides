\documentclass[t]{beamer}
\usepackage{minted}
\usepackage{amsmath}
\usepackage{bm}

\title{Lightning Talk: Using \texttt{auto}}
\author{Edwin Kofler}
\institute{El Camino College Computer Science Club}
\date{March 24, 2023}

\hypersetup{
	colorlinks=true
}
% \setbeameroption{show notes}
\setminted[cpp]{tabsize=3,linenos}


\begin{document}


\frame{\titlepage}


\begin{frame}{Table of Contents}
	\begin{itemize}
		\item Motivation
		\item Trailing Return Types
	\end{itemize}

	\note{
		names are long but googleable
	}
\end{frame}


\begin{frame}{Motivation}
	Why learn extra C++ features?

	\smallskip
	\begin{itemize}
		\item You will read someone else's code than you write your code
		\item You may personally prefer using the feature
		\item Exposure to different features/techniques improves your skills
		\item May be part of an Interview Question
	\end{itemize}

	\note {
		- other people might use different features of C++. it's good to understand them
		- you might like the feature (feels cleaner - chef analogy, save time)
		- the more programming techniques you learn (and understand), the better you will get
		- won't asp about the feature specifically, but may be part of an answer
	}
\end{frame}


\begin{frame}[fragile]{Trailing Return Types}
	\begin{columns}[]
		\begin{column}{.44\textwidth}
			{\large Before}
			\begin{minted}{cpp}
int add_two(int num) {
	int newNum = num + 2;
	return newNum;
}
			\end{minted}
		\end{column}
		\pause
		\begin{column}{.5\textwidth}
			{\large After}
			\begin{minted}{cpp}
auto add_two(int num) -> int {
	int newNum = num + 2;
	return newNum;
}
			\end{minted}
		\end{column}
	\end{columns}
\end{frame}


\begin{frame}[fragile]{Full Example}
	\medbreak
	\begin{minted}{cpp}
#include <iostream>

auto add_two(int num) -> int {
	int newNum = num + 2;
	return newNum;
}

auto main() -> int {
	int num1 = 11;
	int num2 = add_two(num1);
	std::cout << num2 << '\n';
}
\end{minted}
	\note {
		- can even remove the '-> int' part. Return type will be deduced
		- Deduced return types
	}
\end{frame}


\end{document}
