\documentclass[t]{beamer}
\usepackage{minted}

\title{C++ Tips}
\author{Edwin Kofler}
\institute{El Camino College Computer Science Club}
\date{Unknown, 2023}

\hypersetup{
	colorlinks=true
}
% \setbeameroption{show notes}
% \setbeameroption{show only notes}


\begin{document}


\frame{\titlepage}


\begin{frame}{Table Of Contents}
	\begin{enumerate}
		\item Introduction
		\item Compiling with more errors enabled
		\item Using \texttt{nullptr}
		\item Initialization with \texttt{auto}
		\item Trailing return types
		\item Using \texttt{std::span}
	\end{enumerate}
\end{frame}


\begin{frame}{Introduction}
	Have you ever wondered if there is a better way of doing things while writing C++?
	Or are you tired of debugging the same errors over and over again?

	\vspace{10px}
	I'll introduce features that:
	\begin{enumerate}
		\item show you more errors at compile-time (so you don't have to debug them at runtime)
		\item improve code maintainability
		\item save time
		\item you'll come across when reading others' code
	\end{enumerate}

	\vspace{10px}
	Assume C++11
\end{frame}

\begin{frame}[fragile]{Compiling with more errors enabled}
	\begin{minted}[tabsize=3]{cpp}
$ gcc ./main.cpp && ./a.out
	\end{minted}

	\begin{minted}[tabsize=3]{cpp}
$ gcc ./main.cpp -Wall -Wextra -pedantic && ./a.out
	\end{minted}

	\begin{itemize}
		\item VSCode
		\item Visual Studio
		\item Xcode
	\end{itemize}
	\note{
		- Mention other compilers
	}
\end{frame}

\begin{frame}[fragile]{Using \texttt{nullptr}}
	\begin{itemize}
		\item Use \texttt{nullptr} instead of \texttt{NULL}
		\item More errors at compile time
	\end{itemize}

	\begin{minted}[tabsize=3,linenos]{cpp}
#include <iostream>

void print_number(int i) {
	std::cout << "number: " << i << '\n';
}

int main() {
	print_number(40); // ?
	print_number(NULL); // ?
	print_number(nullptr); // ?
}
	\end{minted}

	\note{
		- ask about 40
		- is NULL a number? does this make sense?
		- what does C++ do with NULL?
		- warnings if lucky

		40 = 40
		NULL - 0
		nullptr - compiler error
	}
\end{frame}


\begin{frame}[fragile]{Initialization with \texttt{auto}}
	\begin{columns}[]
		\begin{column}{.5\textwidth}
			{\large Before}
			\begin{minted}[tabsize=3,linenos]{cpp}
int main() {
	int i = 10; // int
	long j = 2L; // long
	float k = 8.0F; // float
	float l = 15.1; // float
}
			\end{minted}
		\end{column}
		\begin{column}{.5\textwidth}
			{\large After}
			\begin{minted}[tabsize=3,linenos]{cpp}
int main() {
	auto i = 10; // ?
	auto j = 2L; // ?
	auto k = 8.0F; // ?
	auto l = 15.1; // ?
}
			\end{minted}
		\end{column}
	\end{columns}

	\begin{itemize}
		\item term: placeholder type specifier
	\end{itemize}

	\note{
		i - int
		j - long
		k - float
		l - double
	}
\end{frame}


\begin{frame}[fragile]{Initialization with \texttt{auto}: Use case 1}

	{\large 1. Initializing a variable with a long type name}

	\begin{minted}[tabsize=3,linenos]{cpp}
class SomeVeryLongClassNameWow {
	...
};

int main() {
	// A
	SomeVeryLongClassNameWow myInstance1 = SomeVeryLongClassNameWow();

	// B
	auto myInstance2 = SomeVeryLongClassNameWow();
}
			\end{minted}

\end{frame}


\begin{frame}[fragile]{Initialization with \texttt{auto}: Use case 2}

	{\large 2. Avoid repetition}

	\begin{minted}[tabsize=3,linenos]{cpp}
#include <vector>

int main() {
	// A
	std::vector<int> myVector1 = std::vector<int>{1, 2, 3};

	// B
	auto myVector2 = std::vector<int>{1, 2, 3};
}
			\end{minted}

\end{frame}


\begin{frame}[fragile]{Trailing Return Types}
	\begin{itemize}
		\item What is the return type of \textit{sayHi}?
	\end{itemize}

	\vspace{10px}
	\begin{columns}[]
		\begin{column}{.48\textwidth}
			{\large Before}
			\begin{minted}[tabsize=3,linenos]{cpp}
void sayHi() {
	std::cout << "Hi!\n";
}

int main() {
	sayHi();
}
			\end{minted}
		\end{column}
		\pause
		\begin{column}{.48\textwidth}
			{\large After}
			\begin{minted}[tabsize=3,linenos]{cpp}
auto sayHi() {
	std::cout << "Hi!\n";
}

int main() {
	sayHi();
}
			\end{minted}
		\end{column}
	\end{columns}

	\vspace{10px}
	\begin{itemize}
		\item Note: Deduction isn't always what you want (\texttt{const}, \texttt{volatile})
	\end{itemize}
	\begin{minted}[tabsize=3,linenos]{cpp}
auto returnOne() {
	int const i = 1;
	return i;
}
	\end{minted}
\end{frame}


\begin{frame}[fragile]{Trailing Return Types (cont.)}
	Usually, you would specify the type

		{\large Can also specify manually:}
	\begin{minted}[tabsize=3,linenos]{cpp}
auto sayHi() -> void {
	std::cout << "Hi!\n";
}

auto returnOne() -> const int {
	int const i = 1;
	return i;
}
	\end{minted}

	\begin{itemize}
		\item term: trailing return types
		\item term: return type deduction
	\end{itemize}
\end{frame}


\begin{frame}[fragile]{Using \texttt{std::span} (Before)}
	How to pass an array of ints to a function?
	\pause

	\begin{minted}[tabsize=2,linenos]{cpp}
		void print_array(int* arr, int arr_len) {
			for (int i = 0; i < arr_len; ++i) {
				std::cout << arr[i] << '\n';
			}
		}

		int main() {
			int my_arr[] = {94, 33, 12};
			int my_len = sizeof(my_arr) / sizeof(my_arr[0]);
			print_array(my_arr, my_len);
		}
	\end{minted}

	\begin{itemize}
		\item Must pass in pointer \textit{and} its length separately
		\item Must use \textit{sizeof}
		\item Array is "converted" into a pointer (pointer decay)
	\end{itemize}


\end{frame}


\begin{frame}[fragile]{Using \texttt{std::span} (After)}

	{\large With \texttt{std::span}}
	\begin{minted}[tabsize=2,linenos]{cpp}
#include <iostream>
#include <span>

void print_arr(std::span<int> arr) {
	for (int i = 0; i < arr.size(); ++i) {
		std::cout << arr[i] << '\n';
	}
}

int main() {
	int* my_arr{94, 33, 12};
	print_arr(my_arr);
}
\end{minted}

	\begin{itemize}
		\item \textbf{Requires C++20}
	\end{itemize}
\end{frame}


\begin{frame}{End}

\end{frame}


\end{document}
